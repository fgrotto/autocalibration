\documentclass[14pt]{article}
\usepackage[T1]{fontenc}
\usepackage{fullpage,graphicx,psfrag,amsmath,amsfonts}
\usepackage[small,bf]{caption}
\usepackage[utf8]{inputenc}
\usepackage[english]{babel}
\usepackage{lipsum}
\usepackage{url}
\usepackage{bm}
\usepackage{float}
\usepackage{physics}
\usepackage{lmodern}
\usepackage[14pt]{extsizes}
\usepackage{enumitem}
\usepackage[left=20mm, right=20mm]{geometry}
\newtheorem{theorem}{Theorem}

\begin{document}
\begin{titlepage}
\begin{center}
    \vspace*{1cm}    
    \textbf{\LARGE Practical Autocalibration}

    \vspace{0.5cm}
    A practical approach and experiments
            
    \vspace{1.5cm}

    \textbf{Filippo Grotto VR460638 \\ Matteo Meneghetti VR469054}

    \vspace{0.5cm}
    Master degree in Computer Engineering for Robotics and Smart Industry

    \vfill
            
    Computer Vision 2021/2022
            
    \vspace{0.8cm}
                
    Department of Computer Science\\
    University of Verona\\
            
\end{center}
\end{titlepage}

\tableofcontents

\begin{thebibliography}{100}
    \addtolength{\leftmargin}{0.2in}
    \setlength{\itemindent}{-0.2in}

    \bibitem{Gherardi10} Riccardo Gherardi and Andrea Fusiello "Practical Autocalibration", ECCV10
\end{thebibliography}

\end{document}